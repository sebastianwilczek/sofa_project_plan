 \section{Introduction}
\label{sec:introduction}
\lhead{\thesection \space Introduction}
This document details the planning of project group 2 of the SOFA module of Fontys University of Applied Sciences. The project plan provides information for the stakeholders of the project, as in how the project will be conducted and what the environmental circumstances of the project are.
\newline
To do so, it will first be detailed how the conduction of the project will be integrated into its environment. This includes the description of the project charter, as well as the change control and how the project is to be closed.
\newline
The scope of the project is defined afterwards. This specifically includes a \textit{Work Breakdown Structure} (WBS) which details the epics that the project can be divided into, as well as what makes up each epic. Deliverables are also defined.
\newline
Furthermore it is described how timing is to be managed during the course of the project. That includes the scheduling of activities and the estimation of the time required to carry these out. While brief, it is also mentioned how costs are handled for the project.
\newline
Quality assurance is covered as well. It is planned who will be checking for quality and what tasks will be executed to do so. It is also mentioned what steps activities go through during sprints to make sure that quality is checked.
\newline
The stakeholders are mentioned next. Among that is a list of all stakeholders as well as a mention of their respective role and interest in the project. It is defined what impact each stakeholder can potentially have on the course of the project and how stakeholders are supposed to be kept informed about changes in the project.
\newline
After stakeholder management the project plan deals with the Human Resource Management of the project. In the Human Resource Management the roles of the stakeholders are listed and managed.
\newline
All the communication is documented in the Communication Management part of the project. Furthermore all the communication vehicles are listed and the stakeholders and their preferred communication vehicle. 
\newline
Penultimate the project plan deals with risk management of the project. Part of the risk management is to identify the risks that can harm the project and calculate their likelihood and probability. Therefore the risks for all three Use-Cases where calculated. 
\newline
Finally the last chapter focus on procurement management. In this chapter all the needed tools that are needed for the project are listed and explained why they are essential for this project. 