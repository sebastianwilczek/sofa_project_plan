\section{Communication Management}
\label{sec:communication}
\lhead{\thesection \space Communication Management}
The overall objective of  Communications Management is to promote the success of a project by meeting the information needs of project stakeholders. The SOFA Connected.Football Communications Management defines the project’s structure and methods of information collection, screening, formatting, and distribution and outline understanding among project teams regarding the actions and processes necessary to facilitate the critical links among people, ideas, and information that are necessary for project success.
 
\subsection{Stakeholder Identification Analysis}

\begin{table}[H]
\centering
\begin{tabular}{|l|l|l|l|l|}
\hline
\cellcolor{gray}Name & \cellcolor{gray}Title & \cellcolor{gray}Contact & \cellcolor{gray}Vehicle \\ \hline
Guido Bruzniak & Product owner & guido.budziak@connected.football & Slack  \\ \hline
Marco Kull & Quality Manager & m.kull@student.fontys.nl & Slack    \\ \hline
Sebastian Wilczek & Software Architect & s.wilczek@student.fontys.nl & Slack   \\ \hline
Patrick Richter & Project Leader & p.richter@student.fontys.nl & Slack \\ \hline
Lucas Gehlen & Scrum Master & l.gehlen@student.fontys.nl & Slack   \\ \hline
Thijs Dorssers & Docent & t.dorssers.fontys.nl & E-Mail   \\ \hline
\end{tabular}
\caption{Stakeholder identification}
\end{table}

\subsection{Communications Vehicles}

\subsubsection{Communication matrix}
The following table shows the communication matrix of the project. 
%
\begin{table}[H]
\centering
\begin{tabular}{|l|l|l|l|l|l|l}
\hline
\cellcolor{gray}Vehicle & \cellcolor{gray}Target & \cellcolor{gray}Purpose & \cellcolor{gray}Frequency & \cellcolor{gray}Owner & \cellcolor{gray}Distribution Vehicle \\ \hline
Project Plan & All stakeholders & 
\begin{tabular}{@{}c@{}}The project plan\\for this project\end{tabular}
& Once & Project Leader & SVN \\ \hline
%& & & & & \\ \hline
\end{tabular}
\caption{Communication matrix}
\end{table}

\subsubsection{Project Meetings}

\begin{table}[H]
\centering
\begin{tabular}{|l|l|l|l|l|}
\hline
\cellcolor{gray}Vehicle & \cellcolor{gray}Frequency & \cellcolor{gray}Internal/External & \cellcolor{gray}Initiator  \\ \hline
Daily & every project day & Internal & Scrum Master     \\ \hline
Customer meeting & every Tuesday & External & Project Leader  \\ \hline
Coach meeting & every Tuesday & Internal & Project Team   \\ \hline
Sprint planing & every Tuesday & Internal & Scrum Master  \\ \hline
Coach meeting & every Tuesday & Internal & Project Team  \\ \hline
\end{tabular}
\caption{Project Meetings}
\end{table}

In this project four meetings were identified. First of all we have the daily scrum meeting where the project team discusses what they have worked on during the most recent project day, what they will do on the current day and if there are any obstacles that stops them from doing their work.
\newline
Then we have the customer meeting in which the project team discuss the progress and obstacles with the customer. The coach meeting informs the coach about the progress and obstacles. Finally in the sprint planning the project team plans the next sprint and decides which issues will be solved in the next sprint, also the past sprint will be discussed and analyzed. 